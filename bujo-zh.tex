\documentclass[a5paper]{article}
\usepackage{xeCJK}
\usepackage{hyperref}
\usepackage{textcomp}
\begin{document}

\paragraph{建议}

\subparagraph{入门}

想要从你的子弹日记中受益最大,一定要阅读笔记本附带的介绍页,以及观看 \href{http://bulletjournal.com}{bulletjournal.com} 上的视频。

\subparagraph{选择最合适的}

这些介绍只不过告诉你了核心方法的框架好让你能快速入门。如果某个方法对你不适用,请放心大胆地调整它,使它更合适。

\subparagraph{简单至上}

虽然我们很鼓励你去个性化自己的子弹日记,不过还是简单些为好。节省些精力好让自己保持高效的状态。

\subparagraph{回顾……再回顾}

即使一天只有 5 分钟,还是要在你早上启动新一天的子弹日记时,以及入睡前回顾自己的子弹日记。这样让你对自己的时间能有个清醒的认识。

\clearpage

\paragraph{介绍}

你现在手上拿着的就是子弹日记笔记本(Bullet Journal)了,它是一个完全客制化的(也是完全海纳百川的)管理体系,可以满足你的需求。它被设计为模块化的,包含可调换的内容让你可以快速获取和管理信息。一本子弹日记可以是你的待办列表、速写本、笔记本、日记,但是最有可能的情况是,它什么都是。

子弹日记包含四种基本概念:快速记录(Rapid Logging)、集合(Collections)、迁移(Migration)、索引(Indexing)。

\clearpage

\paragraph{快速记录}

传统的记笔记、写日程安排的方法都太费时了;登记越复杂,就越费精力,越费精力,它就越容易成为件麻烦事,你就越有可能无法充分利用你的日志,甚至搁置它。快速记录就是解决之道。

快速记录基于简单的符号,或者说基于“子弹”(Bullet)。子弹就是简短的陈述句能让你快速记录事项到你的子弹日记。为了帮你更好地组织条目,他们被分为三类:任务(Tasks)、事件(Events)、笔记(Notes)。

\subparagraph{任务标志}

一项任务使用点 \textbullet 表示,代表任务尚未完成(e.g. “取干洗的衣服”)。任务有四种状态:

\textbf{未完成任务(Task)}

\textbf{已迁移任务(Migrated Task)}已经被迁移到下个月或者加入到集合的任务。

\textbf{已安排任务(Scheduled Task)}规定在将来特定一天完成的任务。可以添加到你的未来记录(Future Log)。

\textbf{已完成任务(Completed Task)}

有些任务需要分步完成。这些步骤——也就是子任务——只需要缩进后列在主任务下即可。只有当子任务全部完成(或者子任务已经失效了),主任务才可以被标记为完成。

\clearpage

\subparagraph{事件标志}

一项事件使用 \textbigcircle 表示。“事件”(Events)是与日期有关联的项目,它可以是计划(e.g. “查理的生日”)或者是已经发生的事件(e.g. “已签订租约”)。在被记录之前就已经完成了的任务也要标记为事件(e.g. “买了飞机票”)。本来被安排的日期已经过期的事件需要被加入到未来记录。

一个事件项,无论它多么私人或者感觉上很繁琐,在快速记录时都应该很客观、简洁。事件“电影之夜”就比“最好的朋友搬到波士顿”更适合。也就是说,只要你已经完成快速记录了,你就可以自由地在下一页详尽地描述它。

\subparagraph{笔记标志}

一条笔记使用短横线\textendash 表示。笔记包含事实、灵感、想法、观察结果。笔记是你想要记住但又不紧急的条目。这个条目在你开会、听讲座、听课时很有用。

\clearpage

\paragraph{建议}

\subparagraph{创建提示符}

提示符(Signifiers)是可以给你的子弹日记增加额外内容的符号。下面列举了一些有用的示例;你也可以随意构想自己用得舒服的提示符。

\subparagraph{优先(Priority)}

使用 \textasteriskcentered 提示符代表任务优先;在每个重要项目的左侧加上提示符,这样在你快速浏览时马上可以找到最重要的事项。

\subparagraph{发掘(Explore)}

使用眼睛符号表示;当有事物需要进一步的研究发掘时使用。

\subparagraph{灵感(Inspiration)}

使用感叹号表示;时常和“笔记”标志一起使用。好主意、个人格言、天才想法从此再也不会遗失!

\subparagraph{创建自己的项目符号}

项目符号和提示符可以轻松满足你的需求,而且你还可以在笔记本的“符号页”(key)定义自己的符号。举个例子,有些人喜欢提前安排事件——尤其是需要自己出席的事件——可以使用下列的会面项目符号:

会面项目符号。

会面已完成项目符号。

\clearpage

\paragraph{集合}

集合(Collections)可以组织管理你的事项。它们用途广泛,可以是购物清单,可以是会议笔记,可以是个人目标。创建集合也很简单:只需要给空白页制定一个主题即可。集合有三种核心类型:月记录(Monthly Log)、日记录(Daily Log)、未来记录(Future Log)。

\subparagraph{月记录}

月记录(Monthly Log)帮你管理——猜对了——你的月计划。它包含一个月历和待办清单。初次设置月记录,你要翻到下一个可用的跨页。左页是你的月历页(Calendar Page);右页是你的任务页(Task Page)。

月历提供给你一个月的概览。这样设置它:将当前月作为标题。现在在月历页标题下方左侧列出当前月的日期,跟在日期后面的是星期的首字母。14 号星期一就是“14 M”。最左边要留些空间方便添加提示符。

你可以使用月历页来记录或安排事件和任务。尽量使得每个事项简洁,因为月历页只是让你大体浏览当月。

任务页用来存放你这个月需要完成的任务以及上个月推迟的任务。

\textbf{建议}

在当月末去设置下一个月的月记录,而不是提早设置。你怎么会知道你需要使用多少页。

\clearpage

使用最适合你的时间框架。有些子弹日记使用者还使用周记录(Weekly Log),在其中管理自己上周推迟的任务,这样的话任务在他们头脑中比较清晰。

\subparagraph{日记录}

日记录(Daily Log)是每天都要使用的。在最上方将当前日期作为你的主题。在全天中,快速记录你的任务、事件、笔记。如果使用不完一整页,就在结束的地方开始明天的记录。

\textbf{建议}

不要提前设置日记录。在你出发前或者睡觉前设置比较好。你也不会知道一天的任务要占用多少笔记本空间。

\subparagraph{未来记录}

这种集合内存放的事项可以是你提前想好要安排进月计划中的,或者是你想某天再处理的。将你的页面按照月份划分来设置你的未来记录(Future Log)。比如说,两条横跨两页的水平线就将页面划分为六个月。

\textbf{建议}

保留一个未来记录页可以用来保存还没有计划的未来任务,比如目标(e.g. “粉刷房屋”)。

你可以在 \href{http://bulletjournal.com}{http://bulletjournal.com} 找到一系列可选方案来规范管理你的未来记录。挑一款最你最心动的。

\clearpage

\paragraph{迁移}

迁移(Migration)是子弹日记的基石。每次你开始你新一个月的计划时,都要回顾上一个月的事项。看看是不是有没有完成的任务?给你完成的任务打上 \texttimes 号,考虑未完成的任务是否还有效。

如果一个任务失效了,就将一整行划掉,包括项目符号。如果一个任务还需要你保持关注,就迁移它:在项目符号 \textbullet 上加一个迁移符号 \textgreater,表示任务已迁移,然后将任务加到新一个月记录的待办页。

你还可以迁移已经安排进日程的任务和事件。如果你正好在设置新一个月的月记录,将你未来记录页中所有安排在本月的任务都迁移过来。如果项目已经被安排了特定日期就写在月历页。

也许将一个事项写来写去有些麻烦,但这是有意义的。这个过程让你停下脚步去思考每个事项。如果你都懒得重新写某个事项,说明它不重要了。那就不要管它了。

迁移的目的是为了分离出值得花费精力的事情,你也可以更清晰地认知自己的模式和爱好,也可以从嘈杂中获得有用的信号。

\clearpage

\paragraph{索引}

这个笔记本的前几页就是索引(Index)。索引是子弹日记内容真正整合的地方。当你开始做子弹日记时,就要在索引页写上你的集合和页数,这样你才可以随时查找翻阅。

连续好几页的集合可以这样写成“主题名:5-10”。

还有的集合可能在你的子弹日记中分散出现,可以写成“主题名:5-10,23,34-39,……”。

索引还可以用来分组其他的事项。如果你拿笔记本来绘画,就创建一个条目叫“绘画”,下面跟着对应的页码。

\paragraph{为什么选择“子弹日记”}

时间是我们拥有的最宝贵的资源。组织管理只是子弹日记的一部分。其真正目的是为了刻意训练你,甄别出最值得花费时间的任务并专注完成的能力。

到 \href{http://bulletjournal.com}{bulletjournal.com} 挖掘更多建议、窍门、方法。

\clearpage

\paragraph{子弹日记官网}

子弹日记有活跃的社区支持,他们为子弹日记的完善添砖加瓦。介绍页包含了最基础的精华方法。想要看更多种多样的方法、建议、窍门,或者你也想出一份力,一定要访问\href{http://bulletjournal.com}{bulletjournal.com}。

\end{document}
